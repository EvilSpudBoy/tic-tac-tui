\documentclass[11pt]{article}
\usepackage[margin=1in]{geometry}
\usepackage[T1]{fontenc}
\usepackage[utf8]{inputenc}
\usepackage{lmodern}
\usepackage{graphicx}
\usepackage{hyperref}
\usepackage{longtable}
\usepackage{xcolor}
\usepackage{fancyhdr}
\hypersetup{colorlinks=true,linkcolor=blue,urlcolor=blue}
\pagestyle{fancy}
\fancyhf{}
\lhead{TicTacTwo User Guide}
\rhead{\thepage}
\setlength{\headheight}{14pt}

\title{TicTacTwo Terminal User Guide}
\author{Project: \texttt{tic-tac-two}}
\date{February 7, 2026}

\begin{document}
\maketitle
\tableofcontents
\newpage

\section{Overview}
TicTacTwo is a terminal-based strategy game inspired by Tic-Tac-Toe, but played on a 5x5 board with a movable active 3x3 grid. You can play human versus AI, AI versus AI, and switch control between human and AI during a match.

\subsection{Core Features}
\begin{itemize}
\item Three startup modes: play as X, play as O, or watch self-play (computer vs computer).
\item Legal move types: \texttt{place}, \texttt{move}, and \texttt{shift}.
\item AI powered by minimax with alpha-beta pruning.
\item Real-time engine analysis panel with principal variations.
\item Scrollable move history with keyboard navigation.
\item Pluggable evaluation functions (default and positional included).
\item Optional strict repetition rule for no-repeat states.
\item Interactive command system during gameplay (\texttt{ai}, \texttt{restart}, \texttt{quit}, and more).
\end{itemize}

\section{Install and Run}
\subsection{Requirements}
\begin{itemize}
\item Node.js 20+
\item npm
\end{itemize}

\subsection{Install and start}
\begin{enumerate}
\item Install dependencies:
\begin{verbatim}
npm install
\end{verbatim}
\item Start the TUI:
\begin{verbatim}
npm start
\end{verbatim}
\end{enumerate}

\subsection{Useful launch options}
\begin{longtable}{p{0.36\textwidth}p{0.58\textwidth}}
\texttt{--engine-depth=<n>} & Search depth for AI move selection (default: 6). \\
\texttt{--multi-pv=<n>} & Number of PV lines shown in analysis panel (default: 3). \\
\texttt{--eval=<name>} & Select evaluation plugin (for example: \texttt{default} or \texttt{positional}). \\
\texttt{--list-evals} & Print available evaluation plugins and exit. \\
\texttt{--self-play} & Start directly in AI vs AI mode. \\
\texttt{--repetition-rule=strict} & Disallow repetition by a matching board plus active-grid state. \\
\end{longtable}

\section{Rules and Game Model}
\subsection{Board and objective}
\begin{itemize}
\item The full board is 5x5.
\item Only the active 3x3 window is playable for placement and movement.
\item Win by forming a three-in-a-row for your symbol in the current position.
\end{itemize}

\subsection{Move types}
\begin{itemize}
\item \textbf{Place}: put your marker on an empty cell inside the active grid.
\item \textbf{Move}: reposition one of your existing markers to another empty cell in the active grid.
\item \textbf{Shift}: move the active 3x3 grid one step in any of the eight directions (up, down, left, right, or diagonally) when legal.
\end{itemize}

\subsection{Piece limits and movement requirements}
\begin{itemize}
\item Each player has exactly four markers total.
\item After placing all four pieces, you must move an existing marker rather than placing new ones.
\item Shifting or moving pieces only becomes legal after placing at least two markers.
\end{itemize}

\subsection{Repetition handling}
\begin{itemize}
\item Default mode avoids immediate move-level backtracking loops.
\item \texttt{--repetition-rule=strict} enforces no repeated full game state, including grid location.
\end{itemize}

\section{Walkthrough 1: Start a Match and Make a Move}
When the app starts, choose your mode:
\begin{itemize}
\item \texttt{X}: you move first.
\item \texttt{O}: AI moves first, then you respond.
\item \texttt{C}: self-play mode.
\end{itemize}

After mode selection, the screen shows:
\begin{itemize}
\item Current board and highlighted active grid.
\item Engine panel with depth, node count, and top PV lines.
\item Move selector with legal moves and a filter prompt.
\end{itemize}

\begin{center}
\includegraphics[width=\textwidth]{docs/user-guide/screens/01_startup.png}

\small Figure 1: Startup prompt, board rendering, engine panel, and move selector.
\end{center}

\section{Walkthrough 2: Use Keyboard Controls During Play}
\subsection{Move selector controls}
\begin{itemize}
\item \texttt{Tab}, \texttt{Up}, \texttt{Down}: cycle through legal moves.
\item Type any text: filter candidate move list.
\item \texttt{Enter}: confirm the currently selected move.
\end{itemize}

\subsection{History controls}
\begin{itemize}
\item \texttt{PgUp} and \texttt{PgDn}: scroll move history.
\item The history panel updates after each move and shows the latest sequence.
\end{itemize}

\subsection{Command shortcuts}
\begin{itemize}
\item \texttt{ai} or \texttt{auto}: let AI take the current turn.
\item \texttt{restart} or \texttt{r}: restart the match.
\item \texttt{exit}, \texttt{quit}, or \texttt{q}: leave the app.
\end{itemize}

\begin{center}
\includegraphics[width=\textwidth]{docs/user-guide/screens/04_controls.png}

\small Figure 2: Interactive controls and command shortcuts available in the TUI.
\end{center}

\section{Walkthrough 3: Observe Self-Play and Grid Shift}
Launch self-play directly:
\begin{verbatim}
npm run play -- --self-play --engine-depth=2 --multi-pv=2
\end{verbatim}

In this mode, the AI alternates moves automatically. This is useful for:
\begin{itemize}
\item Understanding strategy and shift timing.
\item Watching how the active grid changes move legality.
\item Comparing engine lines in real time.
\end{itemize}

The capture below shows a sequence where O executes \texttt{shift left}, which relocates the active grid and changes future legal moves.

\begin{center}
\includegraphics[width=0.92\textwidth]{docs/user-guide/screens/03_selfplay.png}

\small Figure 3: Self-play with engine analysis, history, and an active-grid shift event.
\end{center}

\section{Walkthrough 4: Discover and Select Evaluation Plugins}
List available evaluation plugins:
\begin{verbatim}
npm run play -- --list-evals
\end{verbatim}

Current built-in plugins:
\begin{itemize}
\item \texttt{default}: terminal-only scoring (wins, losses, draws).
\item \texttt{positional}: adds heuristics for threats, center influence, and active-grid presence.
\end{itemize}

Start with a specific plugin:
\begin{verbatim}
npm run play -- --eval=positional --engine-depth=7 --multi-pv=4
\end{verbatim}

\begin{center}
\includegraphics[width=\textwidth]{docs/user-guide/screens/02_plugins.png}

\small Figure 4: Evaluation plugin listing output.
\end{center}

\section{Feature-by-Feature Reference}
\subsection{Gameplay and interaction}
\begin{itemize}
\item Human vs AI and AI vs AI modes.
\item Type-ahead filtering for move commands.
\item Immediate command entry without leaving the move prompt.
\item Restart flow that preserves runtime settings.
\end{itemize}

\subsection{Engine and analysis}
\begin{itemize}
\item Minimax plus alpha-beta pruning.
\item Configurable search depth and Multi-PV count.
\item Per-turn analysis details: nodes, TT hits, cutoffs.
\item Principal variation preview for top candidate lines.
\end{itemize}

\subsection{Extensibility}
\begin{itemize}
\item Plugin registry for custom evaluation strategies.
\item CLI selection by plugin id (\texttt{--eval=<id>}).
\item Discovery endpoint via \texttt{--list-evals}.
\end{itemize}

\section{Development and Testing}
\subsection{Build and test commands}
\begin{verbatim}
npm run build
npm test
npm run coverage
\end{verbatim}

\subsection{Quality notes}
\begin{itemize}
\item The project includes unit tests for core game logic, AI behavior, CLI parsing, and TUI integration paths.
\item Coverage reports are generated under \texttt{coverage/}.
\end{itemize}

\section{Troubleshooting}
\begin{itemize}
\item If terminal rendering looks off, resize the terminal and relaunch.
\item If keys are not recognized, confirm focus is in the terminal window.
\item For strict anti-loop behavior, add \texttt{--repetition-rule=strict}.
\item Use lower depth values if AI turns feel slow on your machine.
\end{itemize}

\section{Quick Command Cheat Sheet}
\begin{verbatim}
npm run play
npm run play -- --self-play
npm run play -- --engine-depth=8 --multi-pv=4
npm run play -- --eval=positional
npm run play -- --list-evals
npm run play -- --repetition-rule=strict
\end{verbatim}

\vfill
\noindent\textit{Generated for the tic-tac-tui repository.}

\end{document}
